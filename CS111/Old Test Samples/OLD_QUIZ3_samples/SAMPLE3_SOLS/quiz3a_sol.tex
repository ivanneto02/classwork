\documentclass[12pt]{article}
\usepackage{fullpage,latexsym,picinpar,amsmath,amsfonts}
%%%%%%%%%%%%%%%%%%%%%%%%%%%%%%%%%%%%%%%%%%%%%%%%%%%%%%%%%%%%%%%%%%%%%%%%%%%%%%%%%%%
%%%%%%%%%%%  LETTERS 
%%%%%%%%%%%%%%%%%%%%%%%%%%%%%%%%%%%%%%%%%%%%%%%%%%%%%%%%%%%%%%%%%%%%%%%%%%%%%%%%%%%

\newcommand{\barx}{{\bar x}}
\newcommand{\bary}{{\bar y}}
\newcommand{\barz}{{\bar z}}
\newcommand{\bart}{{\bar t}}

\newcommand{\bfP}{{\bf{P}}}

%%%%%%%%%%%%%%%%%%%%%%%%%%%%%%%%%%%%%%%%%%%%%%%%%%%%%%%%%%%%%%%%%%%%%%%%%%%%%%%%%%%
%%%%%%%%%%%%%%%%%%%%%%%%%%%%%%%%%%%%%%%%%%%%%%%%%%%%%%%%%%%%%%%%%%%%%%%%%%%%%%%%%%%
                                                                                
\newcommand{\parend}[1]{{\left( #1  \right) }}
\newcommand{\spparend}[1]{{\left(\, #1  \,\right) }}
\newcommand{\angled}[1]{{\left\langle #1  \right\rangle }}
\newcommand{\brackd}[1]{{\left[ #1  \right] }}
\newcommand{\spbrackd}[1]{{\left[\, #1  \,\right] }}
\newcommand{\braced}[1]{{\left\{ #1  \right\} }}
\newcommand{\leftbraced}[1]{{\left\{ #1  \right. }}
\newcommand{\floor}[1]{{\left\lfloor #1\right\rfloor}}
\newcommand{\ceiling}[1]{{\left\lceil #1\right\rceil}}
\newcommand{\barred}[1]{{\left|#1\right|}}
\newcommand{\doublebarred}[1]{{\left|\left|#1\right|\right|}}
\newcommand{\spaced}[1]{{\, #1\, }}
\newcommand{\suchthat}{{\spaced{|}}}
\newcommand{\numof}{{\sharp}}
\newcommand{\assign}{{\,\leftarrow\,}}
\newcommand{\myaccept}{{\mbox{\tiny accept}}}
\newcommand{\myreject}{{\mbox{\tiny reject}}}
\newcommand{\blanksymbol}{{\sqcup}}
                                                                                                                         
\newcommand{\veps}{{\varepsilon}}
\newcommand{\Sigmastar}{{\Sigma^\ast}}
                           
\newcommand{\half}{\mbox{$\frac{1}{2}$}}    
\newcommand{\threehalfs}{\mbox{$\frac{3}{2}$}}   
\newcommand{\domino}[2]{\left[\frac{#1}{#2}\right]}  

%%%%%%%%%%%% complexity classes

\newcommand{\PP}{\mathbb{P}}
\newcommand{\NP}{\mathbb{NP}}
\newcommand{\PSPACE}{\mathbb{PSPACE}}
\newcommand{\coNP}{\textrm{co}\mathbb{NP}}
\newcommand{\DLOG}{\mathbb{L}}
\newcommand{\NLOG}{\mathbb{NL}}
\newcommand{\NL}{\mathbb{NL}}

%%%%%%%%%%% decision problems

\newcommand{\PCP}{\sc{PCP}}
\newcommand{\Path}{\sc{Path}}
\newcommand{\GenGeo}{\sc{Generalized Geography}}

\newcommand{\malytm}{{\mbox{\tiny TM}}}
\newcommand{\malycfg}{{\mbox{\tiny CFG}}}
\newcommand{\Atm}{\mbox{\rm A}_\malytm}
\newcommand{\complAtm}{{\overline{\mbox{\rm A}}}_\malytm}
\newcommand{\AllCFG}{{\mbox{\sc All}}_\malycfg}
\newcommand{\complAllCFG}{{\overline{\mbox{\sc All}}}_\malycfg}
\newcommand{\complL}{{\bar L}}
\newcommand{\TQBF}{\mbox{\sc TQBF}}
\newcommand{\SAT}{\mbox{\sc SAT}}

%%%%%%%%%%%%%%%%%%%%%%%%%%%%%%%%%%%%%%%%%%%%%%%%%%%%%%%%%%%%%%%%%%%%%%%%%%%%%%%%%%%
%%%%%%%%%%%%%%% for homeworks
%%%%%%%%%%%%%%%%%%%%%%%%%%%%%%%%%%%%%%%%%%%%%%%%%%%%%%%%%%%%%%%%%%%%%%%%%%%%%%%%%%%

\newcommand{\student}[2]{%
{\noindent\Large{ \emph{#1} SID {#2} } \hfill} \vskip 0.1in}

\newcommand{\assignment}[1]{\medskip\centerline{\large\bf CS 111 ASSIGNMENT {#1}}}

\newcommand{\duedate}[1]{{\centerline{due {#1}\medskip}}}     

\newcounter{problemnumber}                                                                                 

\newenvironment{problem}{{\vskip 0.1in \noindent
              \bf Problem~\addtocounter{problemnumber}{1}\arabic{problemnumber}:}}{}

\newcounter{solutionnumber}

\newenvironment{solution}{{\vskip 0.1in \noindent
             \bf Solution~\addtocounter{solutionnumber}{1}\arabic{solutionnumber}:}}
				{\ \newline\smallskip\lineacross\smallskip}

\newcommand{\lineacross}{\noindent\mbox{}\hrulefill\mbox{}}

\newcommand{\decproblem}[3]{%
\medskip
\noindent
\begin{list}{\hfill}{\setlength{\labelsep}{0in}
                       \setlength{\topsep}{0in}
                       \setlength{\partopsep}{0in}
                       \setlength{\leftmargin}{0in}
                       \setlength{\listparindent}{0in}
                       \setlength{\labelwidth}{0.5in}
                       \setlength{\itemindent}{0in}
                       \setlength{\itemsep}{0in}
                     }
\item{{{\sc{#1}}:}}
                \begin{list}{\hfill}{\setlength{\labelsep}{0.1in}
                       \setlength{\topsep}{0in}
                       \setlength{\partopsep}{0in}
                       \setlength{\leftmargin}{0.5in}
                       \setlength{\labelwidth}{0.5in}
                       \setlength{\listparindent}{0in}
                       \setlength{\itemindent}{0in}
                       \setlength{\itemsep}{0in}
                       }
                \item{{\em Instance:\ }}{#2}
                \item{{\em Query:\ }}{#3}
                \end{list}
\end{list}
\medskip
}

%%%%%%%%%%%%%%%%%%%%%%%%%%%%%%%%%%%%%%%%%%%%%%%%%%%%%%%%%%%%%%%%%%%%%%%%%%%%%%%%%%%
%%%%%%%%%%%%% for quizzes
%%%%%%%%%%%%%%%%%%%%%%%%%%%%%%%%%%%%%%%%%%%%%%%%%%%%%%%%%%%%%%%%%%%%%%%%%%%%%%%%%%%

\newcommand{\quizheader}{ {\large NAME: \hskip 3in SID:\hfill}
                                \newline\lineacross \medskip }

%\newcommand{\namespace}{ {\large NAME: \hskip 3in SID:\hfill}
%                               \newline\lineacross \medskip }

%%%%%%%%%%%%%%%%%%%%%%%%%%%%%%%%%%%%%%%%%%%%%%%%%%%%%%%%%%%%%%%%%%%%%%%%%%%%%%%%%%%
%%%%%%%%%%%%% for final
%%%%%%%%%%%%%%%%%%%%%%%%%%%%%%%%%%%%%%%%%%%%%%%%%%%%%%%%%%%%%%%%%%%%%%%%%%%%%%%%%%%

\newcommand{\namespace}{\noindent{\Large NAME: \hfill SID:\hskip 1.5in\ }\\\medskip\noindent\mbox{}\hrulefill\mbox{}}



\begin{document}
\smallskip

%%%%%%%%%%%%%%%%%%%%%%%%%%%%%%%%%%%%%%%%%%%%%%%%%%%%%%%%%%%%%%%%%%

\begin{problem}
In the RSA, suppose that Bob chooses $p = 3$ and $q = 43$.
(a) Determine three correct values of the
public exponent $e$. Justify briefly their correctness
(at most 20 words.)

\smallskip
We have
$\phi(n) = (p-1)(q-1) = 2\cdot 42 = 84$.
The value of
$e$ should satisfy: $\gcd(e,84)=1$. Possible solutions are e.g.
$e=5$, $e=11$ or $e=13$.


\bigskip
(b) For one of the $e$'s you selected, compute
the secret exponent $d$. Show your work.

\smallskip
For $e=5$, we compute $d = 5^{-1}\pmod{84}$.
Since $84+1 = 5\cdot 17$, we get $d=17$.
\end{problem}


%%%%%%%%%%%%%%%%%%%%%%%%%%%%%%%%%%%%%%%%%%%%%%%%%%%%%%%%%%%%%%%%%%

\vfill
\eject
\lineacross

\begin{problem}
Grabbits are genetically modified rabbits that live forever and reproduce asexually on a precise schedule: each grabbit gives birth to three grabbits every Wednesday starting two weeks after birth. So if you start with 1 newly born grabbit, after one week you will still only have 1 grabbit. After two weeks you will have 4 grabbits, namely your first grabbit plus its 3 offspring. In general, how many grabbits will you have after $n$ weeks if you start with one newly born grabbit? Set up a recurrence relation for this problem and solve it.


\smallskip
\noindent (a) Let $G_n$ be the number of grabbits after $n$ weeks. Then $G_n$ includes the $G_{n-1}$ grabbits that were alive in week $n-1$ plus the
newly born grabbits. The number of new grabbits is $3G_{n-2}$, because only the grabbits that were around two weeks earlier can have offspring, and each
of them gives birth to $3$ grabbits. So the recurrence relation is:
%
\begin{align*}
G_n &= G_{n-1} + 3G_{n-2}
	\\
G_0 &= 1
	\\
G_1 &= 1
\end{align*}


\smallskip \noindent 
(b) Characteristic polynomial and its roots:
%
\begin{equation*}
x^2 - x - 3 = 0
\end{equation*}
%
so $x_{1,2} = \frac{1\pm \sqrt {13}}{2}$.

\smallskip \noindent 
(c) General form of the solution:
%
\begin{equation*}
\textstyle G_n = \alpha _{1} (\frac{1 + \sqrt {13}}{2})^n + \alpha _{2} (\frac{1 - \sqrt {13}}{2})^n
\end{equation*}


\smallskip \noindent 
(d) Initial condition equations:
%
\begin{align*}
\alpha_1 + \alpha_2 &= 1 \\
\textstyle \alpha_1(\frac{1+\sqrt {13}}{2}) + \alpha_2(\frac{1-\sqrt {13}}{2}) &= 1
\end{align*}
%
and their solution:
%
\begin{equation*}
	\textstyle
\alpha_1 = \frac{1+\sqrt {13}}{2\sqrt{13}} \quad
\alpha_2 = \frac{-1+\sqrt {13}}{2\sqrt{13}}
\end{equation*}


\smallskip \noindent 
(e) Final answer:
The number of grabbits after $n$ weeks is:
%
\begin{align*}
G_n &= \textstyle \frac{1+\sqrt {13}}{2\sqrt{13}} (\frac{1+ \sqrt {13}}{2})^n 
					 + \frac{-1+\sqrt {13}}{2\sqrt{13}} (\frac{1- \sqrt {13}}{2})^n
					 \\
	&= \textstyle \frac{1}{\sqrt{13}} [ (\frac{1+ \sqrt {13}}{2})^{n+1} 
					 -  (\frac{1- \sqrt {13}}{2})^{n+1} ]
\end{align*}

\end{problem}

\end{document}
