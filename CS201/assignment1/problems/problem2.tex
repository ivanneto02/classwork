
\noindent \textbf{Problem 2:} Consider the following control flow graph (CFG) and answer the following questions.


\begin{itemize}
    \item[1.] Find all the EBBs in the CFG;
    \item[2.] Check if any of the following block sets (and their associated edges) may form a region:
    \begin{itemize}
        \item[a)] $\{B, C, D\}$
        \item[b)] $\{B, C, D, E\}$ 
        \item[c)] $\{C, D, E, F\}$
        \item[d)] $\{B, C, D, E, F, G\}$ 
    \end{itemize} 
    \item[3.] Find the dominator set for each basic block;
    \item[4.] Build the dominance tree for the CFG;   
\end{itemize}

\vskip 0.3in

\begin{center}
\begin{tikzpicture}
[
    node distance = 9mm and 14mm,
    nodestyle/.style = {draw,
                        minimum width=5em,
                        minimum height=2em,
                        font=\ttfamily},
    arr/.style       = {very thick,
                       -latex}
]

\node (q0) [nodestyle] {
\textbf{A}
};
\node (q1) [nodestyle, below=of q0] {
\textbf{B}
};

\draw[arr] (q0) -- (q1);

\node (q2) [nodestyle, below=of q1] {
\textbf{C}
};

\draw[arr] (q1) -- (q2);

\node (q3) [nodestyle, below=of q2, xshift=-3em] {
\textbf{D}
};

\node (q4) [nodestyle, below=of q2, xshift=3em] {
\textbf{E}
};

\draw[arr] (q2) -- (q3);
\draw[arr] (q2) -- (q4);

\node (q5) [nodestyle, below=of q3, xshift=3em] {
\textbf{F}
};

\draw[arr] (q1) to [out=180, in=180] (q5);
\draw[arr] (q3) -- (q5);
\draw[arr] (q4) -- (q5);

\node (q6) [nodestyle, below=of q5] {
\textbf{G}
};

\draw[arr] (q6) to [out=0, in=0] (q1);
\draw[arr] (q5) -- (q6);

\node (q7) [nodestyle, below=of q6] {
\textbf{H}
};

\draw[arr] (q6) -- (q7);

\end{tikzpicture}
\end{center}