
\vskip 0.2in

\noindent \textbf{Solution 1):}

\setlength{\arrayrulewidth}{0.5mm}
\renewcommand{\arraystretch}{2.5}

\noindent
\begin{tabularx}{1.0\textwidth}{|X|X|X|X|X|X|X|}
\hline
\textbf{Block} & \textbf{A} & \textbf{B} & \textbf{C} & \textbf{D} & \textbf{E} & \textbf{F} \\
\hline
\textbf{DF} & - & B & E & E & B & - \\
\hline
\end{tabularx}


\vskip 0.3in

\newpage

\noindent \textbf{Solution 2):}

\noindent The first step in inserting the $\phi$-functions is calculating the global names - names
that live across blocks.

\vskip 0.1in

\noindent To do this, I use the following algorithm given in lecture:

\begin{minted}
[
    bgcolor=gray!20,
    baselinestretch=1.2,
    frame=lines,
    framesep=5mm,
]{c}
Globals <- emptyset
for each block B in the CFG
    VarKill <- emptyset
    for each operator x <- y op z in B, in order
        if y not in VarKill then
            Globals <- Globals union {y}
        if z not in varKill then
            Globals <- Globals union {z}
        VarKill <- VarKill union {x}
        Blocks{x} <- Blocks{x} union {B} // defining blocks
\end{minted}

\vskip 0.2in

\noindent Running this algorithm, the following definition blocks are computed:

\noindent
\begin{tabularx}{1.0\textwidth}{
    |X|X|X|X|X|X|X|X|X|
}
\hline
\textbf{name} & a & b & c & f & g & e & m & n \\
\hline
\textbf{DB(s)} & $\{A\}$ & $\{A, F\}$ & $\{A\}$ & $\{B\}$ & $\{CEFB\}$ & $\{C, D\}$ & $\{E\}$ & - \\
\hline
\end{tabularx}

\textit{Note: DB(s) means defining block(s)}

\vskip 0.2in

\noindent Similarly, the following global names are computed

\vskip 0.1in

\noindent Globals = $\{ a, b, c, g, f, m, n\}$

\vskip 0.1in

\textit{Note: I skip the entry block because UEVar does not generate any actual global names.}

\vskip 0.2in

\noindent I use the following algorithm given in the lecture recording to insert
the $\phi$-functions:

\begin{minted}
[
    bgcolor=gray!20,
    baselinestretch=1.2,
    frame=lines,
    framesep=5mm,
]{c}
/* Already done immediately before this */
compute global names Globals and Blocks(x)
for each block B in CFG
    compute DF(B)

for each name x in Globals
    WorkList <- Blocks(x)
    while WorkList != emptyset
        remove block B from WorkList
        for each block B_df in DF(B)
            if B_df has no phi-function for x then
                insert phi-function for x in B_df
                WorkList <- WorkList union {B_df}
\end{minted}

\vskip 0.2in

\noindent I show each global name being inserted into the CFG:

\vskip 0.2in

\noindent For global name $b$:

\vskip 0.1in

\noindent No $\phi$-functions are inserted because both blocks $A$ and $F$ do not have a dominance frontier. Therefore we 
cannot run the inner for loop in the algorithm.

\vskip 0.2in

\noindent For global name $a$:

\vskip 0.1in

\noindent No $\phi$-functions are inserted because block $A$ does not have a dominance frontier. Therefore we 
cannot run the inner for loop in the algorithm.

\vskip 0.2in

\noindent For global name $c$:

\vskip 0.1in

\noindent No $\phi$-functions are inserted because block $A$ does not have a dominance frontier. Therefore we 
cannot run the inner for loop in the algorithm.

\vskip 0.2in

\noindent For global name $g$:

\begin{center}
\begin{tikzpicture}[
    node distance = 9mm and 14mm,
    nodestyle/.style = {draw,
                        minimum width=8em,
                        minimum height=2em,
                        align=left,
                        font=\ttfamily},
    arr/.style       = {very thick,
                        -latex}
]
\node (q0) [nodestyle, label=left:\textbf{A}] {
$a \leftarrow \_a$ \\
$b \leftarrow \_b $ \\
$c \leftarrow a + b$ \\
$b \leftarrow b * c$
};
\node (q1) [nodestyle, below=of q0, label=left:\textbf{B}]    {
\textcolor{red}{$g \leftarrow \phi(g, g)$} \\
$g \leftarrow b * c$ \\
$f \leftarrow b$
};
\node (q2) [nodestyle,below=of q1, xshift=-6em, label=left:\textbf{C}] {
$e \leftarrow a + f$ \\
$g \leftarrow b * c$
};
\draw[arr] (q0) -- (q1);
\draw[arr] (q1) -- (q2);
\node (q3) [nodestyle,below=of q1, xshift=6em, label=left:\textbf{D}] {
$e \leftarrow a + b$ \\
};
\draw[arr] (q1) -- (q3);
\node (q4) [nodestyle,below=of q3, xshift=-6em, label=left:\textbf{E}] {
\textcolor{red}{$g \leftarrow \phi(g, g)$} \\
$m \leftarrow b * c$ \\
$g \leftarrow a$
};
\draw[arr] (q2) -- (q4);
\draw[arr] (q3) -- (q4);
\node (q5) [nodestyle, below=of q4, label=left:\textbf{F}] {
$b \leftarrow g + f$ \\
$g \leftarrow m + n$
};
\draw[arr] (q4) -- (q5);
\draw[arr] (q4) .. controls +(right:5cm) and +(right:5cm) .. (q1)
    node foreach \p in {0, 0.25, ...,1} [sloped, above, pos=\p]{};

\end{tikzpicture}
\end{center}

\vskip 0.2in

\noindent For global name $f$:

\begin{center}
\begin{tikzpicture}[
    node distance = 9mm and 14mm,
    nodestyle/.style = {draw,
                        minimum width=8em,
                        minimum height=2em,
                        align=left,
                        font=\ttfamily},
    arr/.style       = {very thick,
                        -latex}
]
\node (q0) [nodestyle, label=left:\textbf{A}] {
$a \leftarrow \_a$ \\
$b \leftarrow \_b $ \\
$c \leftarrow a + b$ \\
$b \leftarrow b * c$
};
\node (q1) [nodestyle, below=of q0, label=left:\textbf{B}]    {
$g \leftarrow \phi(g, g)$ \\
\textcolor{red}{$f \leftarrow \phi(f, f)$} \\
$g \leftarrow b * c$ \\
$f \leftarrow b$
};
\node (q2) [nodestyle,below=of q1, xshift=-6em, label=left:\textbf{C}] {
$e \leftarrow a + f$ \\
$g \leftarrow b * c$
};
\draw[arr] (q0) -- (q1);
\draw[arr] (q1) -- (q2);
\node (q3) [nodestyle,below=of q1, xshift=6em, label=left:\textbf{D}] {
$e \leftarrow a + b$ \\
};
\draw[arr] (q1) -- (q3);
\node (q4) [nodestyle,below=of q3, xshift=-6em, label=left:\textbf{E}] {
$g \leftarrow \phi(g, g)$ \\
$m \leftarrow b * c$ \\
$g \leftarrow a$
};
\draw[arr] (q2) -- (q4);
\draw[arr] (q3) -- (q4);
\node (q5) [nodestyle, below=of q4, label=left:\textbf{F}] {
$b \leftarrow g + f$ \\
$g \leftarrow m + n$
};
\draw[arr] (q4) -- (q5);
\draw[arr] (q4) .. controls +(right:5cm) and +(right:5cm) .. (q1)
    node foreach \p in {0, 0.25, ...,1} [sloped, above, pos=\p]{};

\end{tikzpicture}
\end{center}

\vskip 0.2in

\noindent For global name $m$:

\begin{center}
\begin{tikzpicture}[
    node distance = 9mm and 14mm,
    nodestyle/.style = {draw,
                        minimum width=8em,
                        minimum height=2em,
                        align=left,
                        font=\ttfamily},
    arr/.style       = {very thick,
                        -latex}
]
\node (q0) [nodestyle, label=left:\textbf{A}] {
$a \leftarrow \_a$ \\
$b \leftarrow \_b $ \\
$c \leftarrow a + b$ \\
$b \leftarrow b * c$
};
\node (q1) [nodestyle, below=of q0, label=left:\textbf{B}]    {
$g \leftarrow \phi(g, g)$ \\
$f \leftarrow \phi(f, f)$ \\
\textcolor{red}{$m \leftarrow \phi(m, m)$} \\
$g \leftarrow b * c$ \\
$f \leftarrow b$
};
\node (q2) [nodestyle,below=of q1, xshift=-6em, label=left:\textbf{C}] {
$e \leftarrow a + f$ \\
$g \leftarrow b * c$
};
\draw[arr] (q0) -- (q1);
\draw[arr] (q1) -- (q2);
\node (q3) [nodestyle,below=of q1, xshift=6em, label=left:\textbf{D}] {
$e \leftarrow a + b$ \\
};
\draw[arr] (q1) -- (q3);
\node (q4) [nodestyle,below=of q3, xshift=-6em, label=left:\textbf{E}] {
$g \leftarrow \phi(g, g)$ \\
$m \leftarrow b * c$ \\
$g \leftarrow a$
};
\draw[arr] (q2) -- (q4);
\draw[arr] (q3) -- (q4);
\node (q5) [nodestyle, below=of q4, label=left:\textbf{F}] {
$b \leftarrow g + f$ \\
$g \leftarrow m + n$
};
\draw[arr] (q4) -- (q5);
\draw[arr] (q4) .. controls +(right:5cm) and +(right:5cm) .. (q1)
    node foreach \p in {0, 0.25, ...,1} [sloped, above, pos=\p]{};

\end{tikzpicture}
\end{center}

\vskip 0.2in

\noindent For global name $n$:

\vskip 0.1in

\noindent Nothing is done here because $n$ does not have any defining blocks.

\vskip 0.2in

\noindent Final CFG:

\vskip 0.1in

\begin{center}
\begin{tikzpicture}[
    node distance = 9mm and 14mm,
    nodestyle/.style = {draw,
                        minimum width=8em,
                        minimum height=2em,
                        align=left,
                        font=\ttfamily},
    arr/.style       = {very thick,
                        -latex}
]
\node (q0) [nodestyle, label=left:\textbf{A}] {
$a \leftarrow \_a$ \\
$b \leftarrow \_b $ \\
$c \leftarrow a + b$ \\
$b \leftarrow b * c$
};
\node (q1) [nodestyle, below=of q0, label=left:\textbf{B}]    {
$g \leftarrow \phi(g, g)$ \\
$f \leftarrow \phi(f, f)$ \\
$m \leftarrow \phi(m, m)$ \\
$g \leftarrow b * c$ \\
$f \leftarrow b$
};
\node (q2) [nodestyle,below=of q1, xshift=-6em, label=left:\textbf{C}] {
$e \leftarrow a + f$ \\
$g \leftarrow b * c$
};
\draw[arr] (q0) -- (q1);
\draw[arr] (q1) -- (q2);
\node (q3) [nodestyle,below=of q1, xshift=6em, label=left:\textbf{D}] {
$e \leftarrow a + b$ \\
};
\draw[arr] (q1) -- (q3);
\node (q4) [nodestyle,below=of q3, xshift=-6em, label=left:\textbf{E}] {
$g \leftarrow \phi(g, g)$ \\
$m \leftarrow b * c$ \\
$g \leftarrow a$
};
\draw[arr] (q2) -- (q4);
\draw[arr] (q3) -- (q4);
\node (q5) [nodestyle, below=of q4, label=left:\textbf{F}] {
$b \leftarrow g + f$ \\
$g \leftarrow m + n$
};
\draw[arr] (q4) -- (q5);
\draw[arr] (q4) .. controls +(right:5cm) and +(right:5cm) .. (q1)
    node foreach \p in {0, 0.25, ...,1} [sloped, above, pos=\p]{};

\end{tikzpicture}
\end{center}

\noindent Note: Although it looks like $e$ should contain a $\phi$-function at block E, $e$ is not a global variable
as it is not utilized in any subsequent blocks. Therefore we do not need a $\phi$-function for $e$ at block $E$.

\newpage

\noindent \textbf{Solution 3):}

\vskip 0.2in

\noindent I start the process of renaming the $\phi$-function variables by building the dominance tree:

\begin{center}
\begin{tikzpicture}[
    node distance = 9mm and 14mm,
    nodestyle/.style = {draw,
                        minimum width=8em,
                        minimum height=2em,
                        align=left,
                        font=\ttfamily},
    arr/.style       = {very thick,
                        -latex}
]

\node (q0) {A};
\node (q1) [below=of q0] {B};
\node (q2) [below=of q1, xshift=-6em] {C};
\node (q3) [below=of q1, xshift=-0em] {D};
\node (q4) [below=of q1, xshift=6em] {E};
\node (q5) [below=of q4] {F};

\draw[arr] (q0) -- (q1);
\draw[arr] (q1) -- (q2);
\draw[arr] (q1) -- (q3);
\draw[arr] (q1) -- (q4);
\draw[arr] (q4) -- (q5);

\end{tikzpicture}
\end{center}

\vskip 0.2in

\noindent To rename the variables, I follow the order of a preorder traversal of the dominance tree:

\vskip 0.1in

\noindent Pre(DomTree) = $\{A, B, C, D, E, F\}$

\vskip 0.2in

\noindent I will be using the following function definitions as provided by the lecture recording:

\vskip 0.2in

\noindent\textbf{\textit{StartRename():}}
\begin{minted}
[
    bgcolor=gray!20,
    baselinestretch=1.2,
    frame=lines,
    framesep=5mm,
]{c}
StartRename():
    for each global name i
        counter[i] <- 0
        stack[i] <- emptyset
    Rename(B_0)
\end{minted}

\vskip 0.2in

\noindent\textbf{\textit{NewName(n):}}
\begin{minted}
[
    bgcolor=gray!20,
    baselinestretch=1.2,
    frame=lines,
    framesep=5mm,
]{c}
i <- counter[n]
push n_i onto stack[n]
counter[n]++
return n_i
\end{minted}

\vskip 0.2in

\noindent\textbf{\textit{Rename(B):}}
\begin{minted}
[
    bgcolor=gray!20,
    baselinestretch=1.2,
    frame=lines,
    framesep=5mm,
]{c}
for each phi-function in B, x <- phi(...)
    rename x as NewName(x)

for each operation "x <- y op z" in B
    rewrite y as top (stack[y])
    rewrite z as top (stack[z])
    rewrite x as NewName(x)

for each successor of B in the CFG
    rewrite appropriate phi parameters

// pre-order traversal on dom tree
for each successor S of B in the dom tree
    Rename(S)

// back to previous scope
for each operation "x <- y op z" in B
    pop(stack[x])

\end{minted}

\noindent I use the above algorithm and show, for each iteration of the preorder traversal, the transformation at each block:

\vskip 0.1in

\noindent For A:

\begin{center}
\begin{tikzpicture}[
    node distance = 9mm and 14mm,
    nodestyle/.style = {draw,
                        minimum width=8em,
                        minimum height=2em,
                        align=left,
                        font=\ttfamily},
    arr/.style       = {very thick,
                        -latex}
]
\node (q0) [nodestyle, color=red, label=left:\textbf{A}] {
$a_0 \leftarrow \_a$ \\
$b_0 \leftarrow \_b $ \\
$c_0 \leftarrow a_0 + b_0$ \\
$b_1 \leftarrow b_0 * c_0$
};
\node (q1) [nodestyle, below=of q0, label=left:\textbf{B}]    {
$g \leftarrow \phi(g, g)$ \\
$f \leftarrow \phi(f, f)$ \\
$m \leftarrow \phi(m, m)$ \\
$g \leftarrow b * c$ \\
$f \leftarrow b$
};
\node (q2) [nodestyle,below=of q1, xshift=-6em, label=left:\textbf{C}] {
$e \leftarrow a + f$ \\
$g \leftarrow b * c$
};
\draw[arr] (q0) -- (q1);
\draw[arr] (q1) -- (q2);
\node (q3) [nodestyle,below=of q1, xshift=6em, label=left:\textbf{D}] {
$e \leftarrow a + b$ \\
};
\draw[arr] (q1) -- (q3);
\node (q4) [nodestyle,below=of q3, xshift=-6em, label=left:\textbf{E}] {
$g \leftarrow \phi(g, g)$ \\
$m \leftarrow b * c$ \\
$g \leftarrow a$
};
\draw[arr] (q2) -- (q4);
\draw[arr] (q3) -- (q4);
\node (q5) [nodestyle, below=of q4, label=left:\textbf{F}] {
$b \leftarrow g + f$ \\
$g \leftarrow m + n$
};
\draw[arr] (q4) -- (q5);
\draw[arr] (q4) .. controls +(right:5cm) and +(right:5cm) .. (q1)
    node foreach \p in {0, 0.25, ...,1} [sloped, above, pos=\p]{};

\end{tikzpicture}
\end{center}

\noindent Since block A is the first block, I renamed each variable to its 0th version. Since no variables
are defined that are used in the parameters for any $\phi$-functions in B, I do not have to rewrite those parameters
in B appropriately.

\vskip 0.1in

\noindent
\begin{tabularx}{1.0\textwidth}{
    |X|X|X|X|X|X|X|X|X|
}
\hline
\textbf{name} & a & b & c & f & g & e & m & n \\
\hline
\textbf{stack} & $\{a_0\}$ & $\{b_0, b_1\}$ & $\{c_0\}$ & $\emptyset$ & $\emptyset$ & $\emptyset$ & $\emptyset$ & $\emptyset$ \\
\hline
\textbf{count} & 1 & 2 & 1 & 0 & 0 & 0 & 0 & 0 \\
\hline
\end{tabularx}

\vskip 0.2in

\noindent For B:

\begin{center}
\begin{tikzpicture}[
    node distance = 9mm and 14mm,
    nodestyle/.style = {draw,
                        minimum width=8em,
                        minimum height=2em,
                        align=left,
                        font=\ttfamily},
    arr/.style       = {very thick,
                        -latex}
]
\node (q0) [nodestyle, color=blue, label=left:\textbf{A}] {
$a_0 \leftarrow \_a$ \\
$b_0 \leftarrow \_b $ \\
$c_0 \leftarrow a_0 + b_0$ \\
$b_1 \leftarrow b_0 * c_0$
};
\node (q1) [nodestyle, color=red, below=of q0, label=left:\textbf{B}]    {
$g_0 \leftarrow \phi(g, g)$ \\
$f_0 \leftarrow \phi(f, f)$ \\
$m_0 \leftarrow \phi(m, m)$ \\
$g_1 \leftarrow b_1 * c_0$ \\
$f_1 \leftarrow b_1$
};
\node (q2) [nodestyle,below=of q1, xshift=-6em, label=left:\textbf{C}] {
$e \leftarrow a + f$ \\
$g \leftarrow b * c$
};
\draw[arr] (q0) -- (q1);
\draw[arr] (q1) -- (q2);
\node (q3) [nodestyle,below=of q1, xshift=6em, label=left:\textbf{D}] {
$e \leftarrow a + b$ \\
};
\draw[arr] (q1) -- (q3);
\node (q4) [nodestyle,below=of q3, xshift=-6em, label=left:\textbf{E}] {
$g \leftarrow \phi(g, g)$ \\
$m \leftarrow b * c$ \\
$g \leftarrow a$
};
\draw[arr] (q2) -- (q4);
\draw[arr] (q3) -- (q4);
\node (q5) [nodestyle, below=of q4, label=left:\textbf{F}] {
$b \leftarrow g + f$ \\
$g \leftarrow m + n$
};
\draw[arr] (q4) -- (q5);
\draw[arr] (q4) .. controls +(right:5cm) and +(right:5cm) .. (q1)
    node foreach \p in {0, 0.25, ...,1} [sloped, above, pos=\p]{};

\node (qx1) [align=left, color=red, fill=white, right=of q1] {
counter[g++] = 0 \\
counter[f++] = 0 \\
counter[m++] = 0 \\
top of stack[b] = $b_1$ \\
top of stack[c] = $c_0$ \\
counter[g++] = 1 \\
counter[f++] = 1
};

\end{tikzpicture}
\end{center}

\vskip 0.1in

\noindent
\begin{tabularx}{1.0\textwidth}{
    |X|X|X|X|X|X|X|X|X|
}
\hline
\textbf{name} & a & b & c & f & g & e & m & n \\
\hline
\textbf{stack} & $\{a_0\}$ & $\{b_0, b_1\}$ & $\{c_0\}$ & $\{f_0, f_1\}$ & $\{g_0, g_1\}$ & $\emptyset$ & $\{m_0\}$ & $\emptyset$ \\
\hline
\textbf{count} & 1 & 2 & 1 & 2 & 2 & 0 & 1 & 0 \\
\hline
\end{tabularx}

\vskip 0.2in

\noindent For C:

\begin{center}
\begin{tikzpicture}[
    node distance = 9mm and 14mm,
    nodestyle/.style = {draw,
                        minimum width=8em,
                        minimum height=2em,
                        align=left,
                        font=\ttfamily},
    arr/.style       = {very thick,
                        -latex}
]
\node (q0) [nodestyle, color=blue, label=left:\textbf{A}] {
$a_0 \leftarrow \_a$ \\
$b_0 \leftarrow \_b $ \\
$c_0 \leftarrow a_0 + b_0$ \\
$b_1 \leftarrow b_0 * c_0$
};
\node (q1) [nodestyle, color=blue, below=of q0, label=left:\textbf{B}]    {
$g_0 \leftarrow \phi(g, g)$ \\
$f_0 \leftarrow \phi(f, f)$ \\
$m_0 \leftarrow \phi(m, m)$ \\
$g_1 \leftarrow b_1 * c_0$ \\
$f_1 \leftarrow b_1$
};
\node (q2) [nodestyle,below=of q1, color=red, xshift=-6em, label=left:\textbf{C}] {
$e_0 \leftarrow a_0 + f_1$ \\
$g_2 \leftarrow b_1 * c_0$
};
\draw[arr] (q0) -- (q1);
\draw[arr] (q1) -- (q2);
\node (q3) [nodestyle,below=of q1, xshift=6em, label=left:\textbf{D}] {
$e \leftarrow a + b$ \\
};
\draw[arr] (q1) -- (q3);
\node (q4) [nodestyle,below=of q3, xshift=-6em, label=left:\textbf{E}] {
\textcolor{red}{$g \leftarrow \phi(g_2, g)$} \\
$m \leftarrow b * c$ \\
$g \leftarrow a$
};
\draw[arr] (q2) -- (q4);
\draw[arr] (q3) -- (q4);
\node (q5) [nodestyle, below=of q4, label=left:\textbf{F}] {
$b \leftarrow g + f$ \\
$g \leftarrow m + n$
};
\draw[arr] (q4) -- (q5);
\draw[arr] (q4) .. controls +(right:5cm) and +(right:5cm) .. (q1)
    node foreach \p in {0, 0.25, ...,1} [sloped, above, pos=\p]{};

\node (qx1) [align=left, color=red, fill=white, left=of q3, xshift=-11em] {
top of stack[a] = $a_0$ \\
top of stack[f] = $f_1$ \\
counter[e++] = 0 \\
top of stack[b] = $b_1$ \\
top of stack[c] = $c_0$
};

\end{tikzpicture}
\end{center}

\vskip 0.1in

\noindent
\begin{tabularx}{1.0\textwidth}{
    |X|X|X|X|X|X|X|X|X|
}
\hline
\textbf{name} & a & b & c & f & g & e & m & n \\
\hline
\textbf{stack} & $\{a_0\}$ & $\{b_0, b_1\}$ & $\{c_0\}$ & $\{f_0, f_1\}$ & $\{g_0, g_1, g_2\}$ & $\{e_0\}$ & $\{m_0\}$ & $\emptyset$ \\
\hline
\textbf{count} & 1 & 2 & 1 & 2 & 3 & 1 & 1 & 0 \\
\hline
\end{tabularx}

\vskip 0.2in

\noindent Since on the dominance tree C is a leaf, we will need to pop all the $x$ in C in the form $x \leftarrow y \text{ op } z$:

\noindent
\begin{tabularx}{1.0\textwidth}{
    |X|X|X|X|X|X|X|X|X|
}
\hline
\textbf{name} & a & b & c & f & g & e & m & n \\
\hline
\textbf{stack} & $\{a_0\}$ & $\{b_0, b_1\}$ & $\{c_0\}$ & $\{f_0, f_1\}$ & $\{g_0, g_1\}$ & $\emptyset$ & $\{m_0\}$ & $\emptyset$ \\
\hline
\textbf{count} & 1 & 2 & 1 & 2 & 2 & 0 & 1 & 0 \\
\hline
\end{tabularx}


\vskip 0.2in

\noindent For D:

\begin{center}
\begin{tikzpicture}[
    node distance = 9mm and 14mm,
    nodestyle/.style = {draw,
                        minimum width=8em,
                        minimum height=2em,
                        align=left,
                        font=\ttfamily},
    arr/.style       = {very thick,
                        -latex}
]
\node (q0) [nodestyle, color=blue, label=left:\textbf{A}] {
$a_0 \leftarrow \_a$ \\
$b_0 \leftarrow \_b $ \\
$c_0 \leftarrow a_0 + b_0$ \\
$b_1 \leftarrow b_0 * c_0$
};
\node (q1) [nodestyle, color=blue, below=of q0, label=left:\textbf{B}]    {
$g_0 \leftarrow \phi(g, g)$ \\
$f_0 \leftarrow \phi(f, f)$ \\
$m_0 \leftarrow \phi(m, m)$ \\
\textcolor{red}{$g_1 \leftarrow b_1 * c_0$} \\
$f_1 \leftarrow b_1$
};
\node (q2) [nodestyle, color=blue, below=of q1, xshift=-6em, label=left:\textbf{C}] {
$e_0 \leftarrow a_0 + f_1$ \\
$g_2 \leftarrow b_1 * c_0$
};
\draw[arr] (q0) -- (q1);
\draw[arr] (q1) -- (q2);
\node (q3) [nodestyle, color=red, below=of q1, xshift=6em, label=left:\textbf{D}] {
$e_0 \leftarrow a_0 + b_1$ \\
};
\draw[arr] (q1) -- (q3);
\node (q4) [nodestyle,below=of q3, xshift=-6em, label=left:\textbf{E}] {
\textcolor{red}{$g \leftarrow \phi(g_2, g_1)$} \\
$m \leftarrow b * c$ \\
$g \leftarrow a$
};
\draw[arr] (q2) -- (q4);
\draw[arr] (q3) -- (q4);
\node (q5) [nodestyle, below=of q4, label=left:\textbf{F}] {
$b \leftarrow g + f$ \\
$g \leftarrow m + n$
};
\draw[arr] (q4) -- (q5);
\draw[arr] (q4) .. controls +(right:5cm) and +(right:5cm) .. (q1)
    node foreach \p in {0, 0.25, ...,1} [sloped, above, pos=\p]{};

\node (qx1) [align=left, color=red, fill=white, right=of q3] {
top of stack[a] = $a_0$ \\
top of stack[b] = $b_1$ \\
counter[e++] = 1
};

\end{tikzpicture}
\end{center}

\vskip 0.1in

\noindent
\begin{tabularx}{1.0\textwidth}{
    |X|X|X|X|X|X|X|X|X|
}
\hline
\textbf{name} & a & b & c & f & g & e & m & n \\
\hline
\textbf{stack} & $\{a_0\}$ & $\{b_0, b_1\}$ & $\{c_0\}$ & $\{f_0, f_1\}$ & $\{g_0, g_1\}$ & $\{e_0\}$ & $\{m_0\}$ & $\emptyset$ \\
\hline
\textbf{count} & 1 & 2 & 1 & 2 & 2 & 1 & 1 & 0 \\
\hline
\end{tabularx}

\vskip 0.2in

\noindent Since on the dominance tree D is a leaf, we will need to pop all the $x$ in C in the form $x \leftarrow y \text{ op } z$:

\noindent
\begin{tabularx}{1.0\textwidth}{
    |X|X|X|X|X|X|X|X|X|
}
\hline
\textbf{name} & a & b & c & f & g & e & m & n \\
\hline
\textbf{stack} & $\{a_0\}$ & $\{b_0, b_1\}$ & $\{c_0\}$ & $\{f_0, f_1\}$ & $\{g_0, g_1\}$ & $\emptyset$ & $\{m_0\}$ & $\emptyset$ \\
\hline
\textbf{count} & 1 & 2 & 1 & 2 & 2 & 0 & 1 & 0 \\
\hline
\end{tabularx}

\vskip 0.2in

\noindent For E:

\begin{center}
\begin{tikzpicture}[
    node distance = 9mm and 14mm,
    nodestyle/.style = {draw,
                        minimum width=8em,
                        minimum height=2em,
                        align=left,
                        font=\ttfamily},
    arr/.style       = {very thick,
                        -latex}
]
\node (q0) [nodestyle, color=blue, label=left:\textbf{A}] {
$a_0 \leftarrow \_a$ \\
$b_0 \leftarrow \_b $ \\
$c_0 \leftarrow a_0 + b_0$ \\
$b_1 \leftarrow b_0 * c_0$
};
\node (q1) [nodestyle, color=blue, below=of q0, label=left:\textbf{B}]    {
\textcolor{red}{$g_0 \leftarrow \phi(g_3, g)$} \\
\textcolor{red}{$f_0 \leftarrow \phi(f_1, f)$} \\
\textcolor{red}{$m_0 \leftarrow \phi(m_1, m)$} \\
$g_1 \leftarrow b_1 * c_0$ \\
\textcolor{red}{$f_1 \leftarrow b_1$}
};
\node (q2) [nodestyle, color=blue, below=of q1, xshift=-6em, label=left:\textbf{C}] {
$e_0 \leftarrow a_0 + f_1$ \\
$g_2 \leftarrow b_1 * c_0$
};
\draw[arr] (q0) -- (q1);
\draw[arr] (q1) -- (q2);
\node (q3) [nodestyle, color=blue, below=of q1, xshift=6em, label=left:\textbf{D}] {
$e_0 \leftarrow a_0 + b_1$ \\
};
\draw[arr] (q1) -- (q3);
\node (q4) [nodestyle, color=red, below=of q3, xshift=-6em, label=left:\textbf{E}] {
$g_2 \leftarrow \phi(g_2, g_1)$ \\
$m_1 \leftarrow b_1 * c_0$ \\
$g_3 \leftarrow a_0$
};
\draw[arr] (q2) -- (q4);
\draw[arr] (q3) -- (q4);
\node (q5) [nodestyle, below=of q4, label=left:\textbf{F}] {
$b \leftarrow g + f$ \\
$g \leftarrow m + n$
};
\draw[arr] (q4) -- (q5);
\draw[arr] (q4) .. controls +(right:5cm) and +(right:5cm) .. (q1)
    node foreach \p in {0, 0.25, ...,1} [sloped, above, pos=\p]{};

\node (qx1) [align=left, color=red, fill=white, right=of q4] {
counter[g++] = 2 \\
top of stack[b] = $b_1$ \\
top of stack[c] = $c_0$ \\
counter[m++] = 1
top of stack[a] = $a_0$ \\
counter[g++] = 3
};

\end{tikzpicture}
\end{center}

\vskip 0.2in

\noindent
\begin{tabularx}{1.0\textwidth}{
    |X|X|X|X|X|X|X|X|X|
}
\hline
\textbf{name} & a & b & c & f & g & e & m & n \\
\hline
\textbf{stack} & $\{a_0\}$ & $\{b_0, b_1\}$ & $\{c_0\}$ & $\{f_0, f_1\}$ & $\{g_0 g_1 g_2 g_3\}$ & $\emptyset$ & $\{m_0, m_1\}$ & $\emptyset$ \\
\hline
\textbf{count} & 1 & 2 & 1 & 2 & 4 & 0 & 2 & 0 \\
\hline
\end{tabularx}

\vskip 0.2in

\noindent For F:

\begin{center}
\begin{tikzpicture}[
    node distance = 9mm and 14mm,
    nodestyle/.style = {draw,
                        minimum width=8em,
                        minimum height=2em,
                        align=left,
                        font=\ttfamily},
    arr/.style       = {very thick,
                        -latex}
]
\node (q0) [nodestyle, color=blue, label=left:\textbf{A}] {
$a_0 \leftarrow \_a$ \\
$b_0 \leftarrow \_b $ \\
$c_0 \leftarrow a_0 + b_0$ \\
$b_1 \leftarrow b_0 * c_0$
};
\node (q1) [nodestyle, color=blue, below=of q0, label=left:\textbf{B}]    {
$g_0 \leftarrow \phi(g_3, g)$ \\
$f_0 \leftarrow \phi(f_1, f)$ \\
$m_0 \leftarrow \phi(m_1, m)$ \\
$g_1 \leftarrow b_1 * c_0$ \\
$f_1 \leftarrow b_1$
};
\node (q2) [nodestyle, color=blue, below=of q1, xshift=-6em, label=left:\textbf{C}] {
$e_0 \leftarrow a_0 + f_1$ \\
$g_2 \leftarrow b_1 * c_0$
};
\draw[arr] (q0) -- (q1);
\draw[arr] (q1) -- (q2);
\node (q3) [nodestyle, color=blue, below=of q1, xshift=6em, label=left:\textbf{D}] {
$e_0 \leftarrow a_0 + b_1$ \\
};
\draw[arr] (q1) -- (q3);
\node (q4) [nodestyle, color=blue, below=of q3, xshift=-6em, label=left:\textbf{E}] {
$g_2 \leftarrow \phi(g_2, g_1)$ \\
$m_1 \leftarrow b_1 * c_0$ \\
$g_3 \leftarrow a_0$
};
\draw[arr] (q2) -- (q4);
\draw[arr] (q3) -- (q4);
\node (q5) [nodestyle, color=red, below=of q4, label=left:\textbf{F}] {
$b_2 \leftarrow g_3 + f_1$ \\
$g_4 \leftarrow m_1 + n$
};
\draw[arr] (q4) -- (q5);
\draw[arr] (q4) .. controls +(right:5cm) and +(right:5cm) .. (q1)
    node foreach \p in {0, 0.25, ...,1} [sloped, above, pos=\p]{};

\node (qx1) [align=left, color=red, fill=white, right=of q5] {
top of stack(g) = $g_3$ \\
top of stack(f) = $f_1$ \\
counter[b++] = 2 \\
top of stack(m) = $m_1$ \\
top of stack(n) = $\emptyset$ \\
counter[g++] = 4
};

\end{tikzpicture}
\end{center}

\vskip 0.2in

\noindent
\begin{tabularx}{1.0\textwidth}{
    |X|X|X|X|X|X|X|X|X|
}
\hline
\textbf{name} & a & b & c & f & g & e & m & n \\
\hline
\textbf{stack} & $\{a_0\}$ & $\{b_0, b_1, b_2\}$ & $\{c_0\}$ & $\{f_0, f_1\}$ & $\{g_0  ... g_4\}$ & $\emptyset$ & $\{m_0, m_1\}$ & $\emptyset$ \\
\hline
\textbf{count} & 1 & 3 & 1 & 2 & 5 & 0 & 2 & 0 \\
\hline
\end{tabularx}

\vskip 0.3in

\noindent The final CFG:

\begin{center}
\begin{tikzpicture}[
    node distance = 9mm and 14mm,
    nodestyle/.style = {draw,
                        minimum width=8em,
                        minimum height=2em,
                        align=left,
                        font=\ttfamily},
    arr/.style       = {very thick,
                        -latex}
]
\node (q0) [nodestyle, label=left:\textbf{A}] {
$a_0 \leftarrow \_a$ \\
$b_0 \leftarrow \_b $ \\
$c_0 \leftarrow a_0 + b_0$ \\
$b_1 \leftarrow b_0 * c_0$
};
\node (q1) [nodestyle, below=of q0, label=left:\textbf{B}]    {
$g_0 \leftarrow \phi(g_3, g)$ \\
$f_0 \leftarrow \phi(f_1, f)$ \\
$m_0 \leftarrow \phi(m_1, m)$ \\
$g_1 \leftarrow b_1 * c_0$ \\
$f_1 \leftarrow b_1$
};
\node (q2) [nodestyle, below=of q1, xshift=-6em, label=left:\textbf{C}] {
$e_0 \leftarrow a_0 + f_1$ \\
$g_2 \leftarrow b_1 * c_0$
};
\draw[arr] (q0) -- (q1);
\draw[arr] (q1) -- (q2);
\node (q3) [nodestyle, below=of q1, xshift=6em, label=left:\textbf{D}] {
$e_0 \leftarrow a_0 + b_1$ \\
};
\draw[arr] (q1) -- (q3);
\node (q4) [nodestyle, below=of q3, xshift=-6em, label=left:\textbf{E}] {
$g_2 \leftarrow \phi(g_2, g_1)$ \\
$m_1 \leftarrow b_1 * c_0$ \\
$g_3 \leftarrow a_0$
};
\draw[arr] (q2) -- (q4);
\draw[arr] (q3) -- (q4);
\node (q5) [nodestyle, below=of q4, label=left:\textbf{F}] {
$b_2 \leftarrow g_3 + f_1$ \\
$g_4 \leftarrow m_1 + n$
};
\draw[arr] (q4) -- (q5);
% \draw[arr] (q4) .. controls +(right:5cm) and +(right:5cm) .. (q1)
%     node foreach \p in {0, 0.25, ...,1} [sloped, above, pos=\p]{};

% \node (qx1) [align=left, color=red, fill=white, right=of q5] {
% top of stack(g) = $g_3$ \\
% top of stack(f) = $f_1$ \\
% counter[b++] = 2 \\
% top of stack(m) = $m_1$ \\
% top of stack(n) = $\emptyset$ \\
% counter[g++] = 4
% };

\end{tikzpicture}
\end{center}